\documentclass[12pt,preprint]{aastex}
 %\documentclass[12pt]{emulateapj}
\usepackage[margin= 1.0in]{geometry}    % See geometry.pdf to learn the layout options. There are lots.
\geometry{letterpaper} % or letter or a5paper or ... etc
\usepackage{float}
\usepackage{amssymb,amsmath}
\usepackage[]{epsfig,graphicx}
\usepackage{color}
\usepackage{verbatim}
\DeclareGraphicsRule{.tif}{png}{.jpg}{`convert Num1 `dirname Num1`/`basename Num1 .tif`.jpg}
\newcommand{\units}[1]{\ensuremath{\, \mathrm{#1}}}
\usepackage{enumitem}
\usepackage{natbib}
\newcommand{\degree}{\ensuremath{^\circ}}
\usepackage[maxfloats=25]{morefloats}
\usepackage[normalem]{ulem}
\usepackage{hyperref}
\usepackage{ amssymb }
\newcommand{\TRANSPOSE}{\ensuremath{T}}

\bibliographystyle{apalike}


\begin{document}
\title{Clustering of Last.fm Music}

 \author{Jordan Rosenblum, Justin Law \& Erin Grand}
 \affil{Data Science Institute, Columbia University, New York, NY 10027}
 
\date{\today}             

\begin{abstract}
We attempt to build a recommendation system for a subset of the Million Song Dataset. We explore various algorithms including matrix factorization, user-based collaborative filtering, and item-based collaborative filtering. In each case we recommend the top 500 songs that our algorithm returns to each user and compare against the testing set of what the user actually listened to. Matrix factorization gives the worst results, only slightly above the popularity baseline, while artist-based popularity and user-based and item-based collaborative filtering methods yield better results.
\end{abstract}

\tableofcontents

\section{Introduction}
In the online world, recommendation systems are becoming increasingly important in various industries including in retail, entertainment, and even dating. 
Because of their popularity and usefulness, there are numerous incentives to implement a good recommendation system. 
In the music industry, companies use recommendation systems to provide a better service to their listeners. 
With the music industry shifting away from a more traditional distribution model of physical CDs or records, 
having an efficient and well performing recommendation system is very important. 

In building a recommendation system, we had to take into account various trade-offs in the type of models we chose to fit. For example, some methods take more information about the user and the song into account (e.g. metadata) while others use a collaborative filtering approach without taking into account user or song characteristic such as age, genre, location, etc.
There is no set similarity measure between two songs, as evident by the fact that songs by the same artist are often very different. As such, there are many different types of algorithms to explore. 

For our project, we explored various song recommendation systems algorithms such as popularity recommendations, artist based recommensations, matrix factorization, and user and item based similarity measures.  
% memory based collaborative filtering

\subsection{Data}
The main source of data for this project was the Million Song Data Set (MSD)  \citep{Bertin-Mahieux2011}. We used the sample data set from the Kaggle competition which contains tuples of user id, song id, and play count.  

The sample data set had $110000$ users, $163206$ songs, and the number of play counts for each user and song pair.
The scale of the data was too large to explore on our computers, so we subset the data down to a size easily handled by an average laptop. 

We choose to subset the data to users who have listened to more than 27 songs and songs which were listened to by more than 22 users. This resulted in a datset which was 10\% the size of the original. Despite subsetting the data, we are still dealing with a matrix of $8130$ users by $11861$ songs, so some of our algorithms will take a fair amount of time to run.

\subsubsection{Data Statistics}

- Sparsity .0016086
- 

SONG PLAY COUNT  \\
mean = 28 \\
max = 35432 \\
min = 1 \\
std = 215.826789\\

USER PLAY COUNT\\
mean = 42\\
max = 1305\\
min = 5\\
std = 53.31547\\


\subsection{Data Analysis}
In order to perform the analysis, we had to fill in training and testing matrices which contained play counts for user and song pairs. We did a random selection of data elements to split between the training (80\%) and testing (20\%) sets. Next, we implemented various algorithms on the dataset in order to recommend songs to users, including matrix factorization, user-based collaborative filtering, and item-based collaborative filtering.

\subsection{Testing}
We tested our algorithms against each other using a mean average precision 
score for each method.  The mean average precision (MAP) calculation came directly from the Kaggle MSD challenge \citep{kaggleMSD}. The score was used to rank the competition entries.

Specifically, we used MAP@500 scores which take into account the first 500 recommendations given to each user. Generally, the score looks at the precision between a list of recommendations for a user and that user's test set. This was done by calculating the average precision at each element in the list of recommended items (i.e. percentage of correct items among first k recommendations) and averaging these for the first 500 recommendations (i.e. for each $k \epsilon {1, ..., 500}$, calculate the precision and average them together).
We then averaged the score over all users. Intuitively, the score looks at the percentage of recommendations that were in-line with the test set of songs actually listened to but also takes order into account. So it is preferable to recommend the songs in the user's test set earlier in the list of 500. The MAP score is similar to the AUC of the ROC curve discussed in class except it takes ordering of the recommendations into account, which is an important factor to consider in a recommendation system.

We used two different baseline benchmarks to compare our MAP scores against. The first was to simply recommend the top 500 most popular songs to every user. This naive method resulted in a MAP score of 0.0138 (top 500 songs based on number of users who listened to the song) or 0.0126 (top 500 songs based on number of plays the song had). The second method was to first recommend a user the most popular songs of artists the user already had already played (but different songs from that artist). If this did not result in enough recommendations, we would then recommend the overall most popular songs as well at the end (similar to baseline method 1). Although this method is conservative in that it does not go beyond the tastes the user already has in the training set, it resulted in a significant improvement in the MAP score to 0.0448.




\section{Algorithms}

\subsection{Matrix Factorization}
The goal of matrix factorization is to use collaborative methods to build a recommendation system for users based on user ratings of objects (in this case songs). This allowed us to recommend certain songs to users based on their listening history and without the need for using content based approaches. Since our data set contains number of plays for a given user and song (rather than rating), we normalized plays for every song on a scale between 0 and 1 and used this as a proxy for rating. 

We then constructed training and testing matrices, of which both have $N_1$ users (rows denoted by $u_i$) and $N_2$ songs (columns denoted by $v_j$). Of course the matrices were very sparse, containing zeros in all entries except for those in which a user (rows of matrix) has listened to a song (columns of matrix). The goal is to factor the training matrix into the product of two matrixes, $U$ and $V$. The matrix $U$ will be $N_1 \times d$ and the matrix $V$ will be $d \times N_2$. We want to learn a low-rank factorization (i.e. we choose $d$) so as to restrict the patterns we see in the rows and columns of our original matrix (e.g. we think a priori that if a user likes 1 top 100 song, the user may also like other top 100 songs). There is subjectivity in picking $d$ but $20$ is a common place to start. In the factorized matrices, the predicted rating will be $\hat{M}_{ij} = u_i^\TRANSPOSE  v_j$ 

Using a coordinate ascent algorithm over 100 iterations, each row ($u_i$) and column ($v_j$) of the training matrix is then updated (equations \ref{eq1} and \ref{eq2}) in order to maximize the log joint likelihood (equation \ref{eq3}). 

\begin{equation}
u_i = \left( \lambda\sigma^2 I + \sum_{j \in \Omega_{u_i}} v_j v_j^\TRANSPOSE \right)^{-1}\left(\sum_{j \in \Omega_{u_i}} M_{ij} v_{j} \right)
\label{eq1}
\end{equation}

\begin{equation}
v_j = \left( \lambda\sigma^2 I + \sum_{i \in \Omega_{v_j}} u_i u_i^\TRANSPOSE  \right)^{-1}\left(\sum_{j \in \Omega_{v_j}} M_{ij} u_{i} \right)
\label{eq2}
\end{equation}

\begin{equation}
\mathcal{L} = \sum_{(i,j) \in \Omega} \frac{1}{2\sigma^2} {|| M_{ij} - u_i^\TRANSPOSE  v_j||}^2 - \sum_{i=1}^{N_1} \frac{\lambda}{2} ||u_i^2 || - \sum_{i=1}^{N_2} \frac{\lambda}{2} ||v_j^2 || + \text{constant}
\label{eq3} 
\end{equation}

\emph{Note: We use a rank 20 matrix for factorization purposes, a $\lambda = 10$, and calculate the variance of the observations for our $\sigma^2$. Also, $\Omega$ is the set of all indices in the matrix which have an observation}.

We keep track of the root mean square error (RMSE) versus the testing set (i.e. how close our prediction is as compared to the actual normalized play count in the testing set) and the log joint likelihood of the training set as a function of iteration (see Figures \ref{fig:rmseplot} and \ref{fig:likeplot}).


%\begin{figure}[htbp] %  figure placement: here, top, bottom, or page
%   \centering
%   \includegraphics[width=4in]{h845BuzFO9CNgAAAABJRU5ErkJggg==.png} 
%   \caption{how close our prediction is as compared to the actual normalized play count in the testing set }
%   \label{fig:rmseplot}
%\end{figure}
%
%\begin{figure}[htbp] %  figure placement: here, top, bottom, or page
%   \centering
%   \includegraphics[width=4in]{C9LAAAAABJRU5ErkJggg==.png} 
%   \caption{log joint likelihood of the training set as a function of iteration }
%   \label{fig:likeplot}
%\end{figure}


%
%\begin{figure}[ht]
%\begin{minipage}[b]{0.45\linewidth}
%\centering
%\includegraphics[width=\textwidth]{h845BuzFO9CNgAAAABJRU5ErkJggg==.png}
%\caption{default}
%\label{fig:figure1}
%\end{minipage}
%\hspace{0.5cm}
%\begin{minipage}[b]{0.45\linewidth}
%\centering
%\includegraphics[width=\textwidth]{C9LAAAAABJRU5ErkJggg==.png}
%\caption{default}
%\label{fig:figure2}
%\end{minipage}
%\end{figure}

%\subsection{Naked Kmeans }
%\subsection{Non-Negative Matrix Factorization}

\subsection{User Based and Item Based Collaborative Filtering}
%\subsubsection{Intro to Memory-Based Collaborative Filtering}
A simpler set of methods for determining which songs to recommend to individual users is by using user- or item/song- based similarity (both also considered collaborative filtering). These only consider whether or not a user listened to a song and does not take into account the play count. Similarity scores between users (i.e. user-based) and songs (i.e. item-based) were calculated using cosine similarity. Intuitively, they work by recommending users songs that other similar users like (user-based) or by recommending songs which are similar to the songs the user has already listened to (item-based).

\subsubsection{User-Based Collaborative Filtering}
For user-based recommendation, we first calculated the similarity score between every pair of users, $u$ and $v$, using the equation below:

\begin{equation}
sim(u,v) = \frac{\text{\# common items}(u, v)}{{\text{\# items}(u)}^{1/2} \times {\text{\# items}(v)}^{1/2}}
\end{equation}

Then, for each user, we looked at each song in the dataset and found all other users who have listened to that song. Next, we added up the similarity score of each of those users, v, with the original user, u, and got the weight for a particular song, $i$:  

$$w_i = \sum_{v \epsilon V} sim(u, v)$$

The sum is a proxy for how likely the user is to like that particular song. We went through that process for all songs and recommended the user the top 500 songs which had the highest scores. This method led to a MAP value of 0.0377.

\subsubsection{Item-Based Collaborative Filtering}
For item-based recommendation, which turned out to be our best algorithm in terms of a MAP value, we first calculated the similarity score between every pair of songs, $i$ and $j$, using the equation below and saved the most similar songs to each song in the process:

\begin{equation}
sim(i,j) = \frac{\text{\# common users}(i, j)}{{\text{\# items}(i)}^{1/2} \times {\text{\# items}(j)}^{1/2}}
\end{equation}

Then, for each user, we found all the songs listened to. Next, we got the most similar songs to each of those songs listened to by the user. If the same similar songs came up multiple times, we added the weights together. In other words, for each song, b, that was found to be similar to one of the songs a that the user listened to, we calculated the score for that song, $b$:

$$w_b = \sum_{a \epsilon A} sim(a, b)$$

Lastly, we sorted the songs by scores and recommended the user the top 500 songs which had the highest scores. This method led to a MAP value of 0.0479.


%\subsection{Artist Based Recommendation}



\section{Conclusion}
After comparing the various algorithms, the simpler models gave us the best results. Our most complicated model, matrix factorization, gave us results barely above the popularity baseline. On the other hand, simply recommending users songs by artists they have already listened to (the artist-based baseline) resulted in a significantly better recommendation system. Our song-based collaborative filtering technique produced the best results despite being a relatively simple algorithm as compared to matrix factorization. Our final results for each technique employed are summarized in the table below. 

\begin{table}[h]
\begin{center}
\begin{tabular}{lc}

\hline
\bf{Algorithm} &  \bf{MAP score}\\ \hline
Artist Based  Baseline  & 0.0448    \\ 
%\hline
Top 500 Songs by Count  Baseline &  0.0138  \\ 
%\hline
Top 500 Songs by Plays  Baseline &  0.0126  \\ 
%\hline
User Based  CF  &  0.0377 \\ 
%\hline
Item Based  CF  &  0.0479 \\ 
%\hline
Matrix Factorization  &   0.0143  \\ 
%\hline
\end{tabular}
\end{center}
\end{table}


\section{Notes}
Sparsity of $8.08199453451\times 10^{-5}$ (needs updating)

%The matrix was too big for our computers to handle the matrix factorization using the whole set of users and songs, so we subset the data. 
%Subset songs by how many times the song was listened to. i.e the popularity of the song
%= new number 
%Subset users by how many songs they've listened to (how many popular song?)
%= new number 


29 - users
34 - songs 
60 \%

Interesting plots:
- Cumulative Distribution of number of songs with a given play count

To Reference a paper: 

The Million Song Dataset Challenge: \citep{McFee:2012:MSD:2187980.2188222} \\
Million Song Dataset Recommendation Project Report: \citep{li2012million} \\
codebook-based scalable music tagging with poisson matrix factorization: \citep{liangcodebook} \\
Matrix Factorization Techniques for Recommender Systems: \citep{koren2009matrix} \\
A Preliminary Study on a Recommender System for the Million Songs Dataset Challenge: \citep{aiolli2013preliminary} \\



\bibliography{msdref}

\small{
\section{Appendix}

\begin{itemize}

  \item Link to our github repository: \url{https://github.com/eringrand/musicanalysis}
  \item The MSD Kaggle challenge: \url{https://www.kaggle.com/c/msdchallenge}

\end{itemize}
}

\end{document} 
