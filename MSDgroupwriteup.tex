% \documentclass[12pt]{emulateapj}
 \documentclass[12pt,preprint]{aastex}
\usepackage[margin= 1.0in]{geometry}    % See geometry.pdf to learn the layout options. There are lots.
\geometry{letterpaper} % or letter or a5paper or ... etc
\usepackage{float}
\usepackage{amssymb,amsmath}
\usepackage[]{epsfig,graphicx}
\usepackage{color}
\usepackage{verbatim}
\DeclareGraphicsRule{.tif}{png}{.jpg}{`convert Num1 `dirname Num1`/`basename Num1 .tif`.jpg}
\newcommand{\units}[1]{\ensuremath{\, \mathrm{#1}}}
\newcommand{\myemail}{egrand@astro.umd.edu}
\usepackage{enumitem}
\usepackage{natbib,twoopt}
\newcommand{\degree}{\ensuremath{^\circ}}
\citestyle{apj}
\bibliographystyle{aa}
\usepackage[maxfloats=25]{morefloats}
\usepackage[normalem]{ulem}
\usepackage{hyperref}

\begin{document}
\title{Clustering of Last.fm Music Artists}

 \author{Jordan Rosenblum, Justin Law \& Erin Grand}
 \affil{Data Science Institute, Columbia University, New York, NY 10027}
 
\date{\today}             

\begin{abstract}
This is the abstract. 
\end{abstract}


\href{http://www.dtic.upf.edu/~ocelma/MusicRecommendationDataset/}{Music Recommendation Dataset}

Sparsity of $8.08199453451\times 10^{-5}$ (needs updating)

The matrix was too big for our computers to handle the matrix factorization using the whole set of users and songs, so we subset the data. 

The full matrix was originally $163206$ songs x $110000$ users. 

Subset songs by how many times the song was listened to. i.e the popularity of the song
= new number 

Subset users by how many songs they've listened to (how many popular song?)
= new number 

SONG PLAY COUNT  \\
mean = 28 \\
max = 35432 \\
min = 1 \\
std = 215.826789\\

USER PLAY COUNT\\
mean = 42\\
max = 1305\\
min = 5\\
std = 53.31547\\


29 - users
34 - songs 

How to split into training and test? 
Random selection by user or element? ELEMENT. 
Need matrixes to be the same size to compare them.



Interesting plots:
- Cumulative Distribution of number of songs with a given play count

\end{document} 